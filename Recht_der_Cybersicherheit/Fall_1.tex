\documentclass[a4paper]{article}
\usepackage[utf8]{inputenc}
\usepackage{scrpage2}
\pagestyle{scrheadings}
\clearscrheadfoot
\ihead{Lukas Koschorke : 2548813}
\ohead{Tutor: Karin Potel}
\begin{document}
\section*{Fall:}
A überfährt Coco, die Katze des B. Er wollte schon immer einmal eine Katze überfahren, außerdem kann er den B sowieso nicht leiden.
\section*{Strafbarkeit des A gutachterlich:}
\subsection*{Obersatz}
A könnte sich, indem er die Katze des B überfahren hat nach § 303 Abs. 1 StGB der Sachbeschädigung strafbar gemacht haben.\\
\subsection*{Prüfung}
Um Sachbeschädigung begangen zu haben, muss es sich bei Coco um eine Sache handeln. Da Coco eine Katze ist sprich ein Tier, handelt es sich nicht um eine Sache, Tiere werden aber nach § 90a BGB wie Sachen behandelt, wenn nichts weiter dabei steht. Coco war offensichtlich eine fremde Katze, da A kein alleiniger Eigentümer der Katze war. Dadurch, dass die Katze überfahren wurde, ist sie nicht mehr am Leben und dadurch trifft auf jeden Fall das Tatbestandsmerkmals des zerstören zu.\\
A hat auf Vorsätzlich gehandelt, denn im Fallbeispiel steht, dass er schon immereinmal eine Katze überfahren wollte.\\
Außerdem hat er keine Rechtsfertigungsgründe, somit handelt A rehtswidrig.\\
Es liegen für A keine Entschuldigungs- oder Schuldausscließungsgründe vor. A handelte Schuldhaft.\\
Der Strafantrag wird nach §303c nur dann gestellt, wenn B einen Strafantrag stellt, da das überfahren einer Katze kein besonderes öffentliches Interresse aufzeigt.
\end{document}